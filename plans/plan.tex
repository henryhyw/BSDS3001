\documentclass[11pt,a4paper]{article}

\usepackage[margin=1in]{geometry}
\usepackage{setspace}
\usepackage{hyperref}
\usepackage{enumitem}
\usepackage{booktabs}
\usepackage{amsmath}

\hypersetup{
    colorlinks=true,
    urlcolor=blue,
    linkcolor=black,
    citecolor=black
}

\setstretch{1.15}

\begin{document}

\title{Implementation Plan:\\
The Social Impacts of the U.S. Data Center Boom}
\author{Detailed Project Implementation Plan}
\date{\today}

\maketitle

\section{Project Structure and Phases}

The project will be implemented in four main phases:

\begin{enumerate}[leftmargin=*]
    \item \textbf{Phase 1: Conceptual design and data scoping}
    \item \textbf{Phase 2: Data acquisition and construction of the state--year panel}
    \item \textbf{Phase 3: Analysis and modeling}
    \item \textbf{Phase 4: Visualization, interpretation, and reporting}
\end{enumerate}

Each phase is described in detail below, with explicit tasks and outputs.

\section{Phase 1: Conceptual Design and Scoping}

\subsection{Refine Research Questions and Theory}

\begin{itemize}[leftmargin=*]
    \item Finalize the set of research questions:
    \begin{itemize}
        \item RQ1: Uneven data center growth across states.
        \item RQ2: Data centers and electricity use/prices.
        \item RQ3: Data centers, emissions, and air quality.
        \item RQ4: Data centers and water stress.
        \item RQ5: Data centers and housing outcomes.
        \item RQ6: Typologies of states via clustering.
    \end{itemize}
    \item Clarify the theoretical lenses:
    \begin{itemize}
        \item Critical infrastructure and socio-technical systems.
        \item Environmental justice (distribution of burdens/benefits).
        \item Political economy of digital infrastructure and AI.
    \end{itemize}
    \item Map out hypothesized causal chains, e.g.:
    \[
    \text{Digital \& AI growth} \rightarrow \text{Data center expansion} \rightarrow
    \text{Electricity demand} \rightarrow \text{Prices \& emissions} \rightarrow
    \text{Community impacts}.
    \]
\end{itemize}

\subsection{Define Unit of Analysis and Time Window}

\begin{itemize}[leftmargin=*]
    \item Unit: U.S.\ state (50 states + District of Columbia).
    \item Time frame: six most recent years with consistent data (e.g.\ 2017--2022).
    \item Dataset size: 51 units $\times$ 6 years = 306 panel observations.
\end{itemize}

\section{Phase 2: Data Acquisition and Construction}

This is the core technical phase, focused on building the integrated dataset.

\subsection{Step 2.1: Create Panel Skeleton}

\begin{itemize}[leftmargin=*]
    \item Construct a base table with variables:
    \begin{itemize}
        \item \texttt{state}, \texttt{state\_fips}, \texttt{year}.
    \end{itemize}
    \item Enumerate all combinations of state and year for 2017--2022.
    \item This table is the merge ``backbone'' for all subsequent joins.
\end{itemize}

\subsection{Step 2.2: Data Center and Digital Sector Data}

\subsubsection*{BLS QCEW -- NAICS 518210}

\begin{itemize}[leftmargin=*]
    \item Download state-level QCEW data for NAICS 518210 for the period 2017--2022.
    \item Key variables:
    \begin{itemize}
        \item \texttt{dc\_emp}: annual average employment.
        \item \texttt{dc\_estab}: number of establishments.
        \item \texttt{dc\_total\_wages}: total annual wages.
        \item \texttt{dc\_avg\_wage}: derived as total wages divided by employment.
    \end{itemize}
    \item Similarly download QCEW totals for all industries to construct:
    \begin{itemize}
        \item \texttt{total\_emp}: total private employment.
        \item \texttt{dc\_emp\_share} = \texttt{dc\_emp / total\_emp}.
    \end{itemize}
\end{itemize}

\subsubsection*{Data Center Footprint -- Open Atlas}

\begin{itemize}[leftmargin=*]
    \item Download the open data center atlas or similar geospatial dataset with facility points and building footprints.
    \item Aggregate by state:
    \begin{itemize}
        \item \texttt{dc\_facilities\_count}: number of data center facilities.
        \item \texttt{dc\_total\_sqft}: sum of data center floor area.
    \end{itemize}
    \item Merge state land area (from Census Gazetteer) to compute:
    \begin{itemize}
        \item \texttt{dc\_sqft\_per\_capita}: footprint per resident.
        \item \texttt{dc\_land\_share}: footprint as a fraction of land area.
    \end{itemize}
    \item Treat these footprint indicators as static across the 2017--2022 panel.
\end{itemize}

\subsection{Step 2.3: Electricity Sales and Prices (EIA)}

\begin{itemize}[leftmargin=*]
    \item Download state-level electricity sales and average price data from EIA for 2017--2022.
    \item Extract variables:
    \begin{itemize}
        \item \texttt{elec\_sales\_total\_mwh}: total sales (all sectors).
        \item \texttt{elec\_sales\_comm\_mwh}: commercial-sector sales.
        \item \texttt{elec\_sales\_ind\_mwh}: industrial-sector sales.
        \item \texttt{price\_all\_cents\_kwh}: average retail price, all sectors.
        \item \texttt{price\_comm\_cents\_kwh}: commercial price.
        \item \texttt{price\_ind\_cents\_kwh}: industrial price.
    \end{itemize}
    \item Later, compute per-capita metrics after merging population data:
    \begin{itemize}
        \item \texttt{elec\_sales\_pc\_mwh} = \texttt{elec\_sales\_total\_mwh / pop}.
    \end{itemize}
\end{itemize}

\subsection{Step 2.4: Emissions (EIA SEDS)}

\begin{itemize}[leftmargin=*]
    \item Download state-level energy-related CO\textsubscript{2} emissions from the State Energy Data System.
    \item Focus on variables:
    \begin{itemize}
        \item \texttt{co2\_power\_mt}: power-sector CO\textsubscript{2} (million metric tons).
        \item Optionally \texttt{co2\_total\_mt}: total CO\textsubscript{2} from all sectors.
    \end{itemize}
    \item Compute per-capita measures:
    \begin{itemize}
        \item \texttt{co2\_power\_pc\_tons} = \texttt{co2\_power\_mt} $\times 10^6 /$ \texttt{pop}.
    \end{itemize}
\end{itemize}

\subsection{Step 2.5: Water Use and Water Stress}

\subsubsection*{USGS Water Use}

\begin{itemize}[leftmargin=*]
    \item Download state-level water-use data (e.g.\ 2015 and 2020).
    \item Extract public supply withdrawals:
    \begin{itemize}
        \item \texttt{public\_withdrawals\_mgd}: million gallons per day.
    \end{itemize}
    \item Compute per-capita withdrawals:
    \begin{itemize}
        \item \texttt{public\_withdrawals\_pc\_gpd} = \texttt{public\_withdrawals\_mgd} $\times 10^6 /$ \texttt{pop}.
    \end{itemize}
    \item Use 2015 values for earlier years and 2020 for later years, or treat the average as a static indicator.
\end{itemize}

\subsubsection*{Baseline Water Stress (Aqueduct)}

\begin{itemize}[leftmargin=*]
    \item Download baseline annual water stress indices for U.S.\ administrative units.
    \item Aggregate to the state level if necessary.
    \item Add:
    \begin{itemize}
        \item \texttt{baseline\_water\_stress}: continuous index, or categorized (low/medium/high).
        \item Optionally \texttt{baseline\_water\_depletion}.
    \end{itemize}
    \item Treat these as static over the 2017--2022 period.
\end{itemize}

\subsection{Step 2.6: Air Quality (EPA AQS)}

\begin{itemize}[leftmargin=*]
    \item Download pre-generated daily PM\textsubscript{2.5} (and optionally ozone) files for 2017--2022.
    \item For each monitor:
    \begin{itemize}
        \item Compute annual average PM\textsubscript{2.5}.
        \item Count days above the standard threshold (e.g.\ 12 $\mu$g/m\textsuperscript{3}).
    \end{itemize}
    \item Aggregate to state-year level:
    \begin{itemize}
        \item \texttt{pm25\_mean\_ugm3}: mean of monitor averages (or population-weighted).
        \item \texttt{pm25\_days\_above\_12}: average days above 12 $\mu$g/m\textsuperscript{3}.
    \end{itemize}
\end{itemize}

\subsection{Step 2.7: Socioeconomic and Housing Indicators}

\subsubsection*{ACS State-Level Indicators}

\begin{itemize}[leftmargin=*]
    \item For each year 2017--2022, collect:
    \begin{itemize}
        \item \texttt{pop}: total population.
        \item \texttt{median\_income}: median household income.
        \item \texttt{median\_gross\_rent}: median gross rent.
        \item \texttt{median\_home\_value}: median owner-occupied home value.
        \item \texttt{rent\_burden\_share}: $\%$ renters spending $\geq 30\%$ of income on rent.
    \end{itemize}
\end{itemize}

\subsubsection*{Unemployment and Broader Digital Sector}

\begin{itemize}[leftmargin=*]
    \item From BLS LAUS:
    \begin{itemize}
        \item \texttt{unemployment\_rate}: state annual unemployment rate.
    \end{itemize}
    \item From QCEW (Information sector, NAICS 51):
    \begin{itemize}
        \item \texttt{info\_emp\_share}: information-sector employment as a share of total employment.
    \end{itemize}
\end{itemize}

\subsection{Step 2.8: Merging and Cleaning}

\begin{itemize}[leftmargin=*]
    \item Sequentially merge all datasets onto the \texttt{state--year} skeleton using consistent state codes and years.
    \item Ensure each merge preserves the panel structure (no duplicate state--year combinations).
    \item Handle missing values due to:
    \begin{itemize}
        \item Data suppression (e.g.\ small QCEW cells) by interpolation or exclusion when rare.
        \item Infrequent measurement (e.g.\ water-use years) by carrying forward/back or averaging.
    \end{itemize}
    \item Standardize key continuous variables (e.g.\ z-scores) for clustering:
    \begin{itemize}
        \item \texttt{dc\_emp\_per\_10k\_pop}, \texttt{elec\_sales\_pc\_mwh}, \texttt{price\_all\_cents\_kwh}, \texttt{co2\_power\_pc\_tons}, \texttt{median\_gross\_rent}, etc.
    \end{itemize}
\end{itemize}

\section{Phase 3: Analysis and Modeling}

\subsection{Step 3.1: Descriptive Statistics and EDA}

\begin{itemize}[leftmargin=*]
    \item Compute summary statistics (mean, median, standard deviation) for all key variables.
    \item Produce correlation matrices to explore relationships among:
    \begin{itemize}
        \item Data center intensity measures.
        \item Electricity use and prices.
        \item Emissions and air quality.
        \item Water stress and withdrawals.
        \item Income, unemployment, housing variables.
    \end{itemize}
    \item Identify outliers and interesting cases (e.g.\ states with very high data center intensity and high water stress).
\end{itemize}

\subsection{Step 3.2: Panel Regression Models}

\subsubsection*{Model Family 1: Data Centers, Electricity Use, and Prices}

\paragraph{Specification}

For state $s$ and year $t$:

\begin{align*}
\text{elec\_sales\_pc\_mwh}_{st}
&= \alpha_s + \gamma_t + \beta_1 \text{dc\_emp\_per\_10k\_pop}_{st} 
+ \beta_2 \log(\text{median\_income}_{st})
+ \beta_3 \text{info\_emp\_share}_{st} \\
&\quad + \beta_4 \text{unemployment\_rate}_{st}
+ \varepsilon_{st},
\end{align*}

where $\alpha_s$ are state fixed effects and $\gamma_t$ are year fixed effects.

A similar model will be estimated for average retail price, $\text{price\_all\_cents\_kwh}_{st}$, as the dependent variable.

\paragraph{Interpretation}

\begin{itemize}[leftmargin=*]
    \item $\beta_1$: association between data center employment intensity and electricity use or prices per capita, controlling for income, digital sector size, and unemployment.
    \item Hypothesis: $\beta_1 > 0$ for both electricity use and prices.
\end{itemize}

\subsubsection*{Model Family 2: Emissions and Air Quality}

\paragraph{Specification}

\begin{align*}
\text{co2\_power\_pc\_tons}_{st}
&= \alpha_s + \gamma_t 
+ \delta_1 \text{dc\_emp\_per\_10k\_pop}_{st}
+ \delta_2 \text{elec\_sales\_pc\_mwh}_{st} \\
&\quad + \delta_3 \text{baseline\_water\_stress}_s
+ \delta_4 \big( \text{dc\_emp\_per\_10k\_pop}_{st} \times \text{baseline\_water\_stress}_s \big)
+ u_{st}.
\end{align*}

\paragraph{Interpretation}

\begin{itemize}[leftmargin=*]
    \item $\delta_1$: direct association between data center intensity and power-sector CO\textsubscript{2} per capita.
    \item $\delta_4$: whether this association is stronger in high water-stress states (interaction).
    \item Hypotheses: $\delta_1 > 0$ and $\delta_4 > 0$.
\end{itemize}

Similarly, we can estimate models with air quality measures (e.g.\ PM\textsubscript{2.5}) as outcomes.

\subsubsection*{Model Family 3: Housing and Community Outcomes}

\paragraph{Specification}

\begin{align*}
\text{median\_gross\_rent}_{st}
&= \alpha_s + \gamma_t 
+ \theta_1 \text{dc\_emp\_per\_10k\_pop}_{st}
+ \theta_2 \log(\text{median\_income}_{st}) \\
&\quad + \theta_3 \text{unemployment\_rate}_{st}
+ v_{st}.
\end{align*}

A parallel model will use \texttt{median\_home\_value} as the dependent variable.

\paragraph{Interpretation}

\begin{itemize}[leftmargin=*]
    \item $\theta_1$: association between data center intensity and housing costs, conditional on income and unemployment.
    \item Hypothesis: $\theta_1 > 0$ in data center hub states.
\end{itemize}

\subsubsection*{Robustness and Diagnostics}

\begin{itemize}[leftmargin=*]
    \item Check multicollinearity among predictors (e.g.\ variance inflation factors).
    \item Test alternative measures of data center intensity (e.g.\ employment share vs.\ facilities count).
    \item Consider simple lagged models (data center intensity at $t-1$ predicting outcomes at $t$).
\end{itemize}

\subsection{Step 3.3: Clustering and Typology}

\begin{itemize}[leftmargin=*]
    \item Construct a state-level feature vector (using mean or final-year values) with:
    \begin{itemize}
        \item \texttt{dc\_emp\_per\_10k\_pop}, \texttt{dc\_facilities\_count}
        \item \texttt{elec\_sales\_pc\_mwh}, \texttt{price\_all\_cents\_kwh}
        \item \texttt{co2\_power\_pc\_tons}, \texttt{baseline\_water\_stress}
        \item \texttt{median\_gross\_rent}, \texttt{median\_income}
    \end{itemize}
    \item Standardize these variables to have mean zero and unit variance.
    \item Apply $k$-means clustering (e.g.\ $k=3$ or $k=4$) and compare inertia and silhouette scores to choose $k$.
    \item Interpret clusters with descriptive tables and maps, e.g.:
    \begin{itemize}
        \item Cluster A: ``data center hubs under stress''
        \item Cluster B: ``energy-intensive, low-digital''
        \item Cluster C: ``low data center, moderate everything''
        \item Cluster D: ``rich digital states with high housing costs''
    \end{itemize}
\end{itemize}

\section{Phase 4: Visualization and Reporting}

\subsection{Step 4.1: Tableau Dashboards}

Design and implement a set of Tableau dashboards corresponding to the main themes:

\subsubsection*{Dashboard 1: Geography of Data Centers}

\begin{itemize}[leftmargin=*]
    \item Choropleth map: data center employment per 10,000 residents by state and year.
    \item Bar chart: top 10 states by data center intensity for a selected year.
    \item Interactive filters: year slider, ability to highlight specific states.
\end{itemize}

\subsubsection*{Dashboard 2: Energy and Emissions}

\begin{itemize}[leftmargin=*]
    \item Line chart: per-capita electricity sales for selected high- and low-intensity states.
    \item Scatterplot: data center intensity vs.\ electricity use/prices, with trend lines.
    \item Scatterplot: data center intensity vs.\ CO\textsubscript{2} per capita.
\end{itemize}

\subsubsection*{Dashboard 3: Water Stress and Air Quality}

\begin{itemize}[leftmargin=*]
    \item Map with dual encoding: baseline water stress and data center intensity.
    \item Scatter with color by water stress category, showing interaction patterns with emissions or PM\textsubscript{2.5}.
\end{itemize}

\subsubsection*{Dashboard 4: Housing and Clusters}

\begin{itemize}[leftmargin=*]
    \item Scatterplot: data center intensity vs.\ rent or home values (and their growth).
    \item Cluster map: states colored by cluster membership.
    \item Tooltip: cluster mean characteristics (energy, emissions, housing).
\end{itemize}

\subsection{Step 4.2: Interpretation and Narrative}

\begin{itemize}[leftmargin=*]
    \item Synthesize regression and clustering results into a narrative:
    \begin{itemize}
        \item Where are data centers expanding the fastest?
        \item How is this linked to energy use, prices, and emissions?
        \item Which states face ``double burdens'' (e.g.\ high data center intensity and high water stress)?
        \item How do housing markets respond in data center hubs?
    \end{itemize}
    \item Connect findings to SDG 7 and SDG 13, and discuss implications for:
    \begin{itemize}
        \item Energy and climate policy.
        \item Water-resource planning.
        \item Land use and housing policy.
        \item Environmental justice and equity.
    \end{itemize}
\end{itemize}

\subsection{Step 4.3: Limitations and Future Work}

\begin{itemize}[leftmargin=*]
    \item Discuss limitations:
    \begin{itemize}
        \item NAICS 518210 is a proxy for data centers and includes other data-processing services.
        \item State-level data may mask within-state spatial inequalities.
        \item Observational design limits causal claims.
        \item Water-use data are infrequent; water stress indices are static.
    \end{itemize}
    \item Suggest extensions:
    \begin{itemize}
        \item County- or metro-level analysis.
        \item Incorporation of detailed tax incentive and land-use data.
        \item Micro-level analysis of employment types and wage distribution.
    \end{itemize}
\end{itemize}

\section{Indicative Timeline}

Assuming a semester-scale project, an example timeline:

\begin{itemize}[leftmargin=*]
    \item Weeks 1--2: Finalize conceptual design and data source list.
    \item Weeks 3--5: Acquire and clean all datasets; construct the state--year panel.
    \item Weeks 6--8: Conduct descriptive analysis and initial regression models.
    \item Weeks 9--10: Refine models, run robustness checks, and perform clustering.
    \item Weeks 11--12: Build Tableau dashboards and draft narrative.
    \item Weeks 13--14: Finalize report, presentation, and visualizations.
\end{itemize}

\section{Summary}

This implementation plan provides a step-by-step roadmap for building a rich, integrated dataset on data centers and social impacts, applying robust social data science methods, and delivering clear visual and written outputs. Following this plan will ensure that the project:

\begin{itemize}[leftmargin=*]
    \item Meets the requirement for more than 300 instances and 15 attributes.
    \item Uses openly available datasets from credible official sources.
    \item Produces both quantitative results and compelling visual storytelling.
    \item Addresses an important, timely topic at the intersection of digital infrastructure, energy, environment, and communities.
\end{itemize}

\end{document}
